\documentclass{exam}
\begin{document}
\usepackage{siunitx}
\begin{center}
  \bfseries\large
  Department of Bitechnology\\
  Biomolecular simulation   BT2123\\
  Fall semester Jan-May 2023\\
  Quiz 1\\
  January 12, 2023

\begin{flushright} \textbf{Maximum Marks: 20} \end{flushright}  \\ 

\begin{flushleft} \textbf{Maximum Time: 30 minutes} \end{flushleft}  
\vspace{5mm}
\fbox{\fbox{\parbox{5.5in}{\centering
If required, please use/write the approximate values of the desired constant}}}
\end{center}
\begin{flushleft} \textbf{Maximum Time: 30 mintuts} \end{flushleft}

\vspace{5mm}

\begin{questions}
	\question What is the main difference between gravitational  and electrostatic forces? Can gravitational or electrostatic forces be negative ? \\  	
	\question What is the MKS unit of electrostatic potential, V. Electron volt eV is unit of which quantity and what is value of 1 eV in MKS units? 	\\
	\question What is the numerical value of Boltzmann constant \\ 
	
		\question Calculate the ratio of gravitational and electrostatic force between a \textit{P} atom of DNA and Na$^{+}$ ion separated by a distance of 10 $\AA$ in vacuum. The masses of P atom and Na$^{+}$ ion are 30.97 and 22.99 amu respectively, while the charges are -2\textit{e} and +1\textit{e} respectively.  
		
		Given that \\ 
		
	1 \textit{e}  = 1.602 \times{10 $^{-19}$ coulomb}. \\
	
	1amu = 1.66 \times{10 $^{-27}$ kg}. \\
	
	permittivity of free space 8.85 \times{10  $^{-12}$ F/m} \\ 
	
	gravitational force constant is 6.67\times{10 $^{-11}$ m $^{3}$ kg$^{-1} s $^{-2}$} \\

\end{questions}
\end{document}
